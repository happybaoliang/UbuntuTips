%\chapter{序论}
\thispagestyle{empty}

\section{装机必备软件及配置}
\subsection{修改默认的系统启动顺序}
运行下面的命令并修改默认的启动序号:

\verb"$ sudo gedit /boot/grub/grub.cfg"

\subsection{配置系统更新源}
(1) 运行下面的命令找开更新源文件:

\verb"$ sudo gedit /etc/apt/sources.list"

(2) 增加下面的内容
\tiny
\begin{verbatim}
#163源
deb http://mirrors.163.com/ubuntu/ precise main restricted
deb-src http://mirrors.163.com/ubuntu/ precise main restricted
deb http://mirrors.163.com/ubuntu/ precise-updates main restricted
deb-src http://mirrors.163.com/ubuntu/ precise-updates main restricted
deb http://mirrors.163.com/ubuntu/ precise universe
deb-src http://mirrors.163.com/ubuntu/ precise universe
deb http://mirrors.163.com/ubuntu/ precise-updates universe
deb-src http://mirrors.163.com/ubuntu/ precise-updates universe
deb http://mirrors.163.com/ubuntu/ precise multiverse
deb-src http://mirrors.163.com/ubuntu/ precise multiverse
deb http://mirrors.163.com/ubuntu/ precise-updates multiverse
deb-src http://mirrors.163.com/ubuntu/ precise-updates multiverse
deb http://mirrors.163.com/ubuntu/ precise-backports main restricted universe multiverse
deb-src http://mirrors.163.com/ubuntu/ precise-backports main restricted universe multiverse
deb http://mirrors.163.com/ubuntu/ precise-security main restricted
deb-src http://mirrors.163.com/ubuntu/ precise-security main restricted
deb http://mirrors.163.com/ubuntu/ precise-security universe
deb-src http://mirrors.163.com/ubuntu/ precise-security universe
deb http://mirrors.163.com/ubuntu/ precise-security multiverse
deb-src http://mirrors.163.com/ubuntu/ precise-security multiverse
deb http://extras.ubuntu.com/ubuntu precise main
deb-src http://extras.ubuntu.com/ubuntu precise main
#sohu源
deb http://mirrors.sohu.com/ubuntu/ precise main restricted
deb-src http://mirrors.sohu.com/ubuntu/ precise main restricted
deb http://mirrors.sohu.com/ubuntu/ precise-updates main restricted
deb-src http://mirrors.sohu.com/ubuntu/ precise-updates main restricted
deb http://mirrors.sohu.com/ubuntu/ precise universe
deb-src http://mirrors.sohu.com/ubuntu/ precise universe
deb http://mirrors.sohu.com/ubuntu/ precise-updates universe
deb-src http://mirrors.sohu.com/ubuntu/ precise-updates universe
deb http://mirrors.sohu.com/ubuntu/ precise multiverse
deb-src http://mirrors.sohu.com/ubuntu/ precise multiverse
deb http://mirrors.sohu.com/ubuntu/ precise-updates multiverse
deb-src http://mirrors.sohu.com/ubuntu/ precise-updates multiverse
deb http://mirrors.sohu.com/ubuntu/ precise-backports main restricted universe multiverse
deb-src http://mirrors.sohu.com/ubuntu/ precise-backports main restricted universe multiverse
deb http://mirrors.sohu.com/ubuntu/ precise-security main restricted
deb-src http://mirrors.sohu.com/ubuntu/ precise-security main restricted
deb http://mirrors.sohu.com/ubuntu/ precise-security universe
deb-src http://mirrors.sohu.com/ubuntu/ precise-security universe
deb http://mirrors.sohu.com/ubuntu/ precise-security multiverse
deb-src http://mirrors.sohu.com/ubuntu/ precise-security multiverse
deb http://extras.ubuntu.com/ubuntu precise main
deb-src http://extras.ubuntu.com/ubuntu precise main
\end{verbatim}
\normalsize

(3) 然后运行下面的命令更新系统信息:

\verb"$ sudo apt-get update"

\subsection{安装UCloner}

\subsection{安装基本编译环境}

\verb"$ sudo apt-get install build-essential"

\subsection{64位Ubuntu安装32位软件}
(1)首先安张32位库

\begin{verbatim}
	$ sudo apt-get install ia32-libs*
	$ sudo apt-get install getlibs 
	#如果没有,到这个地址下http://frozenfox.freehostia.com/cappy/
\end{verbatim}

(2)然后就可以安装32位包,安装时加--force-architecture

(3)然后用getlibs安装依赖库,如
\begin{verbatim}
	$ sudo dpkg -i --force-architecture cairo-dock*.deb
	$ sudo getlibs cairo-dock
\end{verbatim}

\subsection{ICEauthority无法更新的解决办法}
(1)开机过程中出现如下提示:
\begin{verbatim}
1.could not update ICEauthority file:/var/lib/gdm/.ICEauthority
2./usr/lib/libgconf2-4/gconif-sanity-check-2退出状态为256
\end{verbatim}

(2) 解决办法是:

\verb"$ sudo dpkg-reconfigure gdm"

\subsection{解决E: Sub-process /usr/bin/dpkg returned an error code (1)}
(1)先将原有的info目录移走
\begin{verbatim}
    sudo mv /var/lib/dpkg/info /var/lib/dpkg/info.bak //现将info文件夹更名
    sudo mkdir /var/lib/dpkg/info //再新建一个新的info文件夹
    sudo apt-get update
\end{verbatim}

(2)重装出问题的软件
\begin{verbatim}
    apt-get -f install xxx
    sudo mv /var/lib/dpkg/info/* /var/lib/dpkg/info.bak
\end{verbatim}

(3) 执行完上一步操作后会在新的info文件夹下生成一些文件,现将这些文件全部移到info.bak文件夹下
\begin{verbatim}
    sudo rm -rf /var/lib/dpkg/info //把自己新建的info文件夹删掉
    sudo mv /var/lib/dpkg/info.bak /var/lib/dpkg/info //把以前的info文件夹重新改回名字
\end{verbatim}

\subsection{建立Ubuntu本地源}
(1) 用新立得查看apt-mirror,我的机器已经安装,如果没有装,标记安装,应用。

(2) 配置apt-mirror软件,

\verb"$ sudo gedit /etc/apt/mirror.list"

清空mirror.list文件,写入下列内容。
\begin{verbatim}
############# config ##################
set base_path    /media/Software/mirror
set mirror_path  $base_path/mirror
set skel_path    $base_path/skel
set var_path     $base_path/var
set cleanscript $var_path/clean.sh
set defaultarch  <running host architecture>
set postmirror_script $var_path/postmirror.sh
set run_postmirror 0
set nthreads    20
set _tilde 0
############# end config ##############
#64bit deb
deb-amd64 http://mirrors.sohu.com/ubuntu/ precise main restricted universe multiverse
deb-amd64 http://mirrors.sohu.com/ubuntu/ precise-security main restricted universe multiverse
deb-amd64 http://mirrors.sohu.com/ubuntu/ precise-updates main restricted universe multiverse
deb-amd64 http://mirrors.sohu.com/ubuntu/ precise-proposed main restricted universe multiverse
deb-amd64 http://mirrors.sohu.com/ubuntu/ precise-backports main restricted universe multiverse
#32bit deb
deb-i386 http://mirrors.sohu.com/ubuntu/ precise main restricted universe multiverse
deb-i386 http://mirrors.sohu.com/ubuntu/ precise-security main restricted universe multiverse
deb-i386 http://mirrors.sohu.com/ubuntu/ precise-updates main restricted universe multiverse
deb-i386 http://mirrors.sohu.com/ubuntu/ precise-proposed main restricted universe multiverse
deb-i386 http://mirrors.sohu.com/ubuntu/ precise-backports main restricted universe multiverse
#src pack
deb-src http://mirrors.sohu.com/ubuntu/ precise main restricted universe multiverse
deb-src http://mirrors.sohu.com/ubuntu/ precise-security main restricted universe multiverse
deb-src http://mirrors.sohu.com/ubuntu/ precise-updates main restricted universe multiverse
deb-src http://mirrors.sohu.com/ubuntu/ precise-proposed main restricted universe multiverse
clean http://mirrors.sohu.com/ubuntu
\end{verbatim}

说明:上面内容大意是使用20线程,把服务器上precise版的ubuntu 32/64位软件源的main、restricted、universe、 multiverse几个部分镜像到本地。本地默认存放软件源的文件夹是/media/Software/mirror,请保证至少有120G的剩余空间。

(3) 运行apt-mirror元件,开始镜像。

\verb" $ sudo apt-mirror"

之后你可以干的就是一段不太短的时间的等待。同步过程可以中断,关闭终端就可以。下次你想再同步的时候重新运行sudo apt-mirror就可以。

(4) 备份并打开软件源配置文件,
\begin{verbatim}
$ sudo cp /etc/apt/source.list /etc/apt/source.list_httpsource
$ sudo gedit /etc/apt/source.list
\end{verbatim}

清空文件,写入下列内容,
\begin{verbatim}
deb file:///media/Software/mirror/mirror/mirrors.sohu.com/ubuntu/ precise main restricted
deb file:///media/Software/mirror/mirror/mirrors.sohu.com/ubuntu/ precise-updates main restricted
deb file:///media/Software/mirror/mirror/mirrors.sohu.com/ubuntu/ precise universe
deb file:///media/Software/mirror/mirror/mirrors.sohu.com/ubuntu/ precise-updates universe
deb file:///media/Software/mirror/mirror/mirrors.sohu.com/ubuntu/ precise multiverse
deb file:///media/Software/mirror/mirror/mirrors.sohu.com/ubuntu/ precise-updates multiverse
\end{verbatim}

保存,退出。好了,打开新立得,刷新。开始本地高速安装吧。

\subsection{Ubuntu无法进入系统的解决办法}
ubuntu 12.04进入恢复模式以后,文件系统是只读的,可以通过下面的方法修改。

\verb"# mount /dev/sda1 / -o rw,remount"

如果在登陆界面输入密码后只是闪一下又回到登陆界面的,可以将本用户目录下的.Xauthority文件删除。

\subsection{ubuntu清理系统垃圾}
linux不会产生无用垃圾文件,但是在升级缓存中,linux不会自动删除这些文件。

(1) 非常有用的清理命令:
\begin{verbatim} 
$ sudo apt-get autoclean
$ sudo apt-get clean
$ sudo apt-get autoremove
\end{verbatim}

这三个命令主要清理升级缓存以及无用包的。

(2) 清理opera firefox的缓存文件:
\begin{verbatim}
$ ls ~/.opera/cache4
$ ls ~/.mozilla/firefox/*.default/Cache
\end{verbatim}

(3) 清理Linux下孤立的包:

图形界面下我们可以用:gtkorphan
\verb"$ sudo apt-get install gtkorphan -y"

终端命令下我们可以用:deborphan
\verb"$ sudo apt-get install de borphan -y"

(4) 卸载tracker

这个东西一般我只要安装ubuntu就会第一删掉tracker 他不仅会产生大量的cache文件而且还会影响开机速度。所以在新得利里面删掉就行。

(5) 删除多余的内核:

打开终端敲命令:
\verb"# dpkg --get-selections|grep linux"

有image的就是内核文件

删除老的内核文件:

sudo apt-get remove 内核文件名 (例如:linux-image-2.6.27-2-generic)

内核删除,释放空间了,应该能释放130-140M空间。

最后不要忘了看看当前内核:uname -a

\subsection{安装文件的默认打开方式}
可以用Ubuntu Tweak中的“文件类型管理器”改

\subsection{安装字体}

\subsection{调节CPU频率}
\begin{verbatim}
#cpufrequtils--调节CPU频率
#监视cpu频率:右键单击面板,选择“添加到面板”,里面找到“cpu频率范围监视器”。
sudo apt-get install cpufrequtils --force-yes -y
sudo cpufreq-set -g ondemand
# 执行cpufreq-info可看到CPU所支持的模式,大致有如下几种:
# powersave,是无论如何都只会保持最低频率的所谓“省电”模式;
# userspace,是自定义频率时的模式,这个是当你设定特定频率时自动转变的;
# ondemand,一有cpu计算任务,立即达到最大频率,执行完毕立即回到最低频率
# conservative,保守模式(默认),一般选择这个,会自动在频率上下限调整;
# performance,顾名思义只注重效率,无论如何一直保持以最大频率运行。 
#监视CPU温度:
#添加到面板的项的名字叫“Hardware sensors monitor"
sudo apt-get install sensors-applet --force-yes -y
\end{verbatim}

\subsection{安装NTFS读写支持} 
安装完后,点击“应用程序” - “系统工具” - “NTFS写入支持配置程序”输入密码,选中对内部设备的读写支持,和外部设备的读写支持,系统会自动扫描你硬盘上的NTFS分区,并重新挂载,这样,你的NTFS分区就能在Feisty下完美读写了!
\verb"$ sudo apt-get install ntfs-config --force-yes -y"

\subsection{安装dkms和wine}
\verb"$ sudo apt-get install dkms wine"

\subsection{安装CCSM以启动Ubuntu桌面特效和3D加速效果}
\verb"$ sudo apt-get install compizconfig-settings-manager"

\subsection{安装Ubuntu Restricted Extras}
安装Ubuntu Restricted Extras软件包后,我们就可以播放mp3,avi,Flash等。

\verb"$ sudo apt-get install ubuntu-restricted-extras"

\subsection {解决在virtualbox中无法进入共享目录}
\verb"$ sudo ln -f -s /opt/VBoxGuestAdditions-4.3.10/lib/VBoxGuestAdditions/mount.vboxsf /sbin/mount.vboxsf"

